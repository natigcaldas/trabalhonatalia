\documentclass[]{article}
\usepackage{lmodern}
\usepackage{amssymb,amsmath}
\usepackage{ifxetex,ifluatex}
\usepackage{fixltx2e} % provides \textsubscript
\ifnum 0\ifxetex 1\fi\ifluatex 1\fi=0 % if pdftex
  \usepackage[T1]{fontenc}
  \usepackage[utf8]{inputenc}
\else % if luatex or xelatex
  \ifxetex
    \usepackage{mathspec}
  \else
    \usepackage{fontspec}
  \fi
  \defaultfontfeatures{Ligatures=TeX,Scale=MatchLowercase}
\fi
% use upquote if available, for straight quotes in verbatim environments
\IfFileExists{upquote.sty}{\usepackage{upquote}}{}
% use microtype if available
\IfFileExists{microtype.sty}{%
\usepackage{microtype}
\UseMicrotypeSet[protrusion]{basicmath} % disable protrusion for tt fonts
}{}
\usepackage[margin=1in]{geometry}
\usepackage{hyperref}
\hypersetup{unicode=true,
            pdftitle={Títulos Constitucionais},
            pdfauthor={Natália Grassano Caldas},
            pdfborder={0 0 0},
            breaklinks=true}
\urlstyle{same}  % don't use monospace font for urls
\usepackage{longtable,booktabs}
\usepackage{graphicx,grffile}
\makeatletter
\def\maxwidth{\ifdim\Gin@nat@width>\linewidth\linewidth\else\Gin@nat@width\fi}
\def\maxheight{\ifdim\Gin@nat@height>\textheight\textheight\else\Gin@nat@height\fi}
\makeatother
% Scale images if necessary, so that they will not overflow the page
% margins by default, and it is still possible to overwrite the defaults
% using explicit options in \includegraphics[width, height, ...]{}
\setkeys{Gin}{width=\maxwidth,height=\maxheight,keepaspectratio}
\IfFileExists{parskip.sty}{%
\usepackage{parskip}
}{% else
\setlength{\parindent}{0pt}
\setlength{\parskip}{6pt plus 2pt minus 1pt}
}
\setlength{\emergencystretch}{3em}  % prevent overfull lines
\providecommand{\tightlist}{%
  \setlength{\itemsep}{0pt}\setlength{\parskip}{0pt}}
\setcounter{secnumdepth}{0}
% Redefines (sub)paragraphs to behave more like sections
\ifx\paragraph\undefined\else
\let\oldparagraph\paragraph
\renewcommand{\paragraph}[1]{\oldparagraph{#1}\mbox{}}
\fi
\ifx\subparagraph\undefined\else
\let\oldsubparagraph\subparagraph
\renewcommand{\subparagraph}[1]{\oldsubparagraph{#1}\mbox{}}
\fi

%%% Use protect on footnotes to avoid problems with footnotes in titles
\let\rmarkdownfootnote\footnote%
\def\footnote{\protect\rmarkdownfootnote}

%%% Change title format to be more compact
\usepackage{titling}

% Create subtitle command for use in maketitle
\newcommand{\subtitle}[1]{
  \posttitle{
    \begin{center}\large#1\end{center}
    }
}

\setlength{\droptitle}{-2em}

  \title{Títulos Constitucionais}
    \pretitle{\vspace{\droptitle}\centering\huge}
  \posttitle{\par}
    \author{Natália Grassano Caldas}
    \preauthor{\centering\large\emph}
  \postauthor{\par}
      \predate{\centering\large\emph}
  \postdate{\par}
    \date{17 de janeiro de 2019}


\begin{document}
\maketitle

\section{Resumo}\label{resumo}

Quais são as restrições enfrentadas pelo Supremo Tribunal Federal (STF)
que ocasionam a não inclusão de seus processos em pauta de julgamento?
Como os fatores externos podem influenciar no julgamento das ações
diretas de inconstitucionalidade (ADIs)? A Corte vem demonstrando, ao
longo dos últimos vinte anos, uma tendência cooperativa com os
interesses do Poder Executivo, quando acionada pelos mecanismos de
controle de constitucionalidade (revisão judicial), seja confirmando a
constitucionalidade da legislação, seja ainda simplesmente deixando
informalmente de julgar um grande número de conflitos. Os dados
originais coletados das decisões tomadas pelo STF no julgamento de ações
diretas de inconstitucionalidade apontam na direção da autorrestrição, e
essa respectiva abstenção exclui os freios constitucionais, modifica as
relações federativas regionais e confere certo grau de liberdade ao
legislador diante da ausência de perspectiva de controle judicial de
suas atividades. Nesse sentido, uma amostra aleatória de ADINs será
colhida do sítio eletrônico do STF e submetida à análise estatística por
regressão logística com o objetivo de explicar o comportamento
abstensivo judicial.

\section{Introdução}\label{introducao}

O presente trabalho é fruto de uma inquietação que visa revisitar
algumas abordagens sobre o Supremo Tribunal Federal: por que razões um
volume considerável de processos do controle de constitucionalidade
concentrado permanece inerte durante vários anos sem qualquer decisão
(provisória ou definitiva) sobre o litígio que foi submetido à Suprema
Corte?

O STF possui um comportamento padrão altamente abstensivo em relação ao
julgamento das ADINs. Esse comportamento omissivo é evidenciado em
58,4\% dos casos (22,1\% rejeitadas por vícios de forma e 36,3\%
aguardando julgamento) -- (Barbosa; Gomes Neto; Carvalho; Santos, 2009).
É ressaltado, também, o relevante segmento de 39,26\% de ADINs que estão
enquadradas entre aquelas que aguardam julgamento (Thamy Pogrebinschi,
2011).

Em números mais recentes, do universo total de 5.680 ações diretas de
inconstitucionalidade distribuídas até o dia 7 de abril de 2017, uma
parcela de 1.912 processos (34,69\%) permanecia aguardando julgamento.
Essa aparente estabilidade da fração de processos silenciosamente
excluída da apreciação da jurisdição constitucional, no âmbito do STF,
dá indícios do exercício de seletividade quanto ao julgamento dos
conflitos, fomentando indagações quanto às situações que estariam
associadas a maiores ou menores chances de resolução dos litígios
constitucionais ou de longos períodos de espera por um pronunciamento
acerca da alegada inconstitucionalidade da norma impugnada.

Em uma ADIN, o STF exerce diretamente o seu papel de tribunal
constitucional. O julgamento dessa ação, ao contrário das outras, tem
efeitos erga omnes, ou seja, seus resultados atingem todos os indivíduos
da determinada população. Além disso, uma ADIN deve ser obrigatoriamente
julgada pelo Plenário e requer a participação de todos os ministros
existentes, enriquecendo ainda mais a análise pretendida. Essas são
razões que explicam a popularidade do seu uso na literatura empírica e,
também, no propósito dessa pesquisa.

A análise teórica do comportamento judicial e da sua contextualização
histórica e política consegue explicar a judicialização da política por
meio do aumento expressivo das ações judiciais, entendendo esse enorme
aumento processual como uma forma de participação da sociedade civil. O
suporte para essa afirmação se encontra nos dados empíricos coletados: o
número crescente de ações diretas de inconstitucionalidade (ADINs) no
Supremo Tribunal Federal (STF) atesta o quanto o judiciário brasileiro
vem sendo ``convocado'' a participar das decisões políticas no Brasil.

A investigação deste fenômeno institucional oferece um locus
interessante e inovador para o conhecimento do comportamento judicial,
especialmente quanto à opção expressa ou tácita pelo não exercício da
atividade judicial, isto é, pela autorrestrição.

Trata-se de um comportamento que se apresenta contraditório em relação
ao desenho constitucional brasileiro, cujas disposições determinam o
dever de apreciação jurisdicional de todas as questões submetidas ao
Judiciário (non liquet), sendo a recusa formal ou informal ao exercício
de poder que lhe foi delegado pela Constituição um fenômeno que ainda
demanda respostas substantivas. Sem dúvidas, a expansão do Judiciário, a
judicialização da política e todas as suas implicações são fenômenos que
tomaram conta do final do século XX.

A integração dos tribunais no contexto político gerou, de fato,
alterações significativas no planejamento de instituições políticas.
Porém, o funcionamento das cortes judiciais e seu papel no regime
político atual tem sido muito pouco estudado pela Ciência Política
brasileira. Apesar do papel que uma instituição como o Supremo Tribunal
Federal cumpre no cenário político Brasileiro de hoje, ainda se sabe
pouco acerca dos mecanismos específicos pelos quais esses atores
influenciam ou até determinam o resultado de decisões relativas a
fenômenos políticos. Diante disso, é possível reafirmar a relevância do
estudo empírico -- no qual essa pesquisa se dedica - sobre o
comportamento dos Ministros do Supremo Tribunal e as variáveis que o
afetam, de forma a integrar essa instituição na análise do processo
político decisório nacional.

Quando se trata de entender o comportamento judicial, notadamente, o que
influencia no processo de tomada de decisões -- seja a decisão judicial
expressa, seja a silenciosa decisão de não levar uma questão a
julgamento -- a literatura oferece, entre outros, o modelo estratégico,
ao qual essa pesquisa é dedicada. Nesse modelo positivo, os juízes se
deparam com restrições aos seus objetivos relacionadas com a reação dos
seus pares em um tribunal, com a manifestação de atores externos ou,
ainda, com a reação da opinião pública quando vão julgar uma ação.

Diante desse contexto, o objetivo geral da pesquisa é justamente
verificar se os fatores estratégicos ligados às ações diretas de
inconstitucionalidade que ainda aguardam julgamento influenciam
diretamente no tempo em que estas permanecem sem ser julgadas. A
hipótese levantada é que os relatores seriam seletivos quanto ao que
incluir ou não em pauta de julgamento a partir dos atores envolvidos e
das consequências concretas possíveis que possam vir de uma futura
decisão. A confirmação ou a negação desta hipótese trará informações
relevantes sobre o comportamento do Supremo Tribunal Federal, mais
precisamente sobre a existência de uma seletividade na formação da pauta
de julgamento.

É importante salientar que, seguindo o mesmo raciocínio realizado por
Taylor (2008), está sendo considerado como julgado todo processo de
controle concentrado em que a liminar foi concedida.

Mediante o emprego de ferramentas estatísticas (análise por regressão
logística), essa pesquisa, como mencionado, busca testar a existência,
ou inexistência, de associações entre o não julgamento das ADINs e a
presença de uma série de variáveis categóricas, extraídas das
informações dos respectivos processos e representativas do modelo
estratégico. Para tanto, pretende-se colher uma amostra aleatória de
processos, proporcionalmente distribuídos na série temporal compreendida
entre 1989 e 2017, do universo de mais de 5.000 ADINs distribuídas desde
a promulgação da Constituição Federal de 1988 até aquele momento. Dessa
forma, a pesquisa vai tornar possível a identificação dos fatores que
podem influenciar os julgadores, notadamente os Ministros do Supremo
Tribunal Federal, a não exercer a jurisdição constitucional sobre
determinados conflitos, especialmente quanto à silenciosa decisão de não
submeter as questões a julgamento.

\section{Análise dos dados}\label{analise-dos-dados}

O objetivo das análises será verificar se os temas das Ações Diretas
de Inconstitucionalidade que ainda não foram julgadas influenciam
diretamente no tempo em que estas permanecem sem serem incluídas em
pauta de julgamento. Para tanto, iremos analisar o conjunto de dados
descritivamente e ajustaremos um modelo aos dados para determinar se há
uma influência estatisticamente significativa do título de uma ADIN no
tempo que ela permanece fora de pauta.

\section{Análise Descritiva}\label{analise-descritiva}

\subparagraph{Título I}\label{tatulo-i}

\begin{longtable}[]{@{}lrrrrrrrr@{}}
\toprule
& N & Média & DesvioPadrão & Mínimo & 1Quartil & Mediana & 3Quartil &
Máximo\tabularnewline
\midrule
\endhead
0 & 573 & 2230.672 & 1818.102 & 14 & 817.00 & 1764.0 & 3277.00 &
8592\tabularnewline
1 & 106 & 2802.142 & 1994.000 & 29 & 1242.25 & 2497.5 & 4058.25 &
7450\tabularnewline
\bottomrule
\end{longtable}

\begin{center}\includegraphics{Temas_Constitucionais__2__files/figure-latex/titulo I-1} \end{center}

\subparagraph{Título II}\label{tatulo-ii}

\begin{longtable}[]{@{}lrrrrrrrr@{}}
\toprule
& N & Média & DesvioPadrão & Mínimo & 1Quartil & Mediana & 3Quartil &
Máximo\tabularnewline
\midrule
\endhead
0 & 448 & 2349.973 & 1838.452 & 14 & 925.75 & 1875.5 & 3336.75 &
8592\tabularnewline
1 & 231 & 2261.532 & 1894.419 & 29 & 757.00 & 1866.0 & 3289.50 &
7939\tabularnewline
\bottomrule
\end{longtable}

\begin{center}\includegraphics{Temas_Constitucionais__2__files/figure-latex/titulo II-1} \end{center}

\subparagraph{Título III}\label{tatulo-iii}

\begin{longtable}[]{@{}lrrrrrrrr@{}}
\toprule
& N & Média & DesvioPadrão & Mínimo & 1Quartil & Mediana & 3Quartil &
Máximo\tabularnewline
\midrule
\endhead
0 & 307 & 2157.528 & 1707.344 & 14 & 765.00 & 1876.0 & 3119.5 &
8374\tabularnewline
1 & 372 & 2453.874 & 1963.735 & 22 & 919.25 & 1858.5 & 3612.0 &
8592\tabularnewline
\bottomrule
\end{longtable}

\begin{center}\includegraphics{Temas_Constitucionais__2__files/figure-latex/titulo III-1} \end{center}

\subparagraph{Título IV}\label{tatulo-iv}

\begin{longtable}[]{@{}lrrrrrrrr@{}}
\toprule
& N & Média & DesvioPadrão & Mínimo & 1Quartil & Mediana & 3Quartil &
Máximo\tabularnewline
\midrule
\endhead
0 & 424 & 2220.106 & 1714.425 & 22 & 899 & 1802 & 3209.5 &
8592\tabularnewline
1 & 255 & 2485.792 & 2064.793 & 14 & 748 & 2068 & 3694.0 &
8374\tabularnewline
\bottomrule
\end{longtable}

\begin{center}\includegraphics{Temas_Constitucionais__2__files/figure-latex/titulo IV-1} \end{center}

\subparagraph{Título V}\label{tatulo-v}

\begin{longtable}[]{@{}lrrrrrrrr@{}}
\toprule
& N & Média & DesvioPadrão & Mínimo & 1Quartil & Mediana & 3Quartil &
Máximo\tabularnewline
\midrule
\endhead
0 & 629 & 2311.892 & 1899.880 & 14 & 803.0 & 1771.0 & 3369.00 &
8592\tabularnewline
1 & 50 & 2420.440 & 1196.089 & 79 & 1364.5 & 2474.5 & 3181.25 &
4901\tabularnewline
\bottomrule
\end{longtable}

\begin{center}\includegraphics{Temas_Constitucionais__2__files/figure-latex/titulo V-1} \end{center}

\subparagraph{Título VI}\label{tatulo-vi}

\begin{longtable}[]{@{}lrrrrrrrr@{}}
\toprule
& N & Média & DesvioPadrão & Mínimo & 1Quartil & Mediana & 3Quartil &
Máximo\tabularnewline
\midrule
\endhead
0 & 496 & 2359.192 & 1944.322 & 14 & 849.75 & 1802 & 3462 &
8592\tabularnewline
1 & 183 & 2213.350 & 1595.503 & 29 & 884.50 & 2048 & 3209 &
7415\tabularnewline
\bottomrule
\end{longtable}

\begin{center}\includegraphics{Temas_Constitucionais__2__files/figure-latex/titulo VI-1} \end{center}

\subparagraph{Título VII}\label{tatulo-vii}

\begin{longtable}[]{@{}lrrrrrrrr@{}}
\toprule
& N & Média & DesvioPadrão & Mínimo & 1Quartil & Mediana & 3Quartil &
Máximo\tabularnewline
\midrule
\endhead
0 & 609 & 2294.212 & 1891.154 & 14 & 807.0 & 1775.0 & 3318.0 &
8592\tabularnewline
1 & 70 & 2543.243 & 1518.274 & 400 & 1304.5 & 2709.5 & 3382.5 &
6774\tabularnewline
\bottomrule
\end{longtable}

\begin{center}\includegraphics{Temas_Constitucionais__2__files/figure-latex/titulo VII-1} \end{center}

\subparagraph{Título VIII}\label{tatulo-viii}

\begin{longtable}[]{@{}lrrrrrrrr@{}}
\toprule
& N & Média & DesvioPadrão & Mínimo & 1Quartil & Mediana & 3Quartil &
Máximo\tabularnewline
\midrule
\endhead
0 & 609 & 2294.212 & 1891.154 & 14 & 807.0 & 1775.0 & 3318.0 &
8592\tabularnewline
1 & 70 & 2543.243 & 1518.274 & 400 & 1304.5 & 2709.5 & 3382.5 &
6774\tabularnewline
\bottomrule
\end{longtable}

\begin{center}\includegraphics{Temas_Constitucionais__2__files/figure-latex/titulo VIII-1} \end{center}

\subparagraph{Título IX}\label{tatulo-ix}

\begin{longtable}[]{@{}lrrrrrrrr@{}}
\toprule
& N & Média & DesvioPadrão & Mínimo & 1Quartil & Mediana & 3Quartil &
Máximo\tabularnewline
\midrule
\endhead
0 & 660 & 2323.535 & 1856.027 & 14 & 876 & 1876 & 3326.75 &
8592\tabularnewline
1 & 19 & 2193.105 & 1928.975 & 145 & 805 & 1567 & 3205.50 &
6111\tabularnewline
\bottomrule
\end{longtable}

\begin{center}\includegraphics{Temas_Constitucionais__2__files/figure-latex/titulo IX-1} \end{center}

\subsubsection{Modelos Estatísticos}\label{modelos-estatasticos}

Agora iremos ajustar um modelo de regressão para o tempo fora de pauta
cujas variáveis explicativas serão a presença ou não de cada
título. Começamos ajustando um Modelo de Regressão Linear Normal.
Pela análise de resíduos vemos que este modelo não está bem
ajustado. De fato, analisando-se o gráfico QQ-plot dos resíduos
padronizados pelos quantis teóricos, tem-se que os pontos não estão
bem ajustados sobre a reta diagonal, apresentando caudas acima da reta.
Isso é evidência de que os erros não apresentam distribuição
Normal, e por isso não seguem a suposição do modelo, de forma que o
modelo não está bem ajustado.

\begin{longtable}[]{@{}lrrrrr@{}}
\toprule
& Df & Sum Sq & Mean Sq & F value & Pr(\textgreater{}F)\tabularnewline
\midrule
\endhead
TI & 1 & 2.921306e+07 & 29213055.84 & 8.5682894 &
0.0035368\tabularnewline
TII & 1 & 2.425909e+06 & 2425909.18 & 0.7115275 &
0.3992387\tabularnewline
TIII & 1 & 9.603444e+06 & 9603443.91 & 2.8167230 &
0.0937537\tabularnewline
TIV & 1 & 6.796689e+06 & 6796689.00 & 1.9934922 &
0.1584412\tabularnewline
TV & 1 & 2.700235e+06 & 2700234.93 & 0.7919882 &
0.3738195\tabularnewline
TVI & 1 & 2.911356e+04 & 29113.56 & 0.0085391 & 0.9264021\tabularnewline
TVII & 1 & 5.545047e+06 & 5545046.87 & 1.6263813 &
0.2026470\tabularnewline
TVIII & 1 & 8.831452e+04 & 88314.52 & 0.0259030 &
0.8721861\tabularnewline
TIX & 1 & 1.232210e+05 & 123220.97 & 0.0361411 &
0.8492820\tabularnewline
Residuals & 669 & 2.280914e+09 & 3409438.49 & NA & NA\tabularnewline
\bottomrule
\end{longtable}

\begin{center}\includegraphics{Temas_Constitucionais__2__files/figure-latex/modNormal-1} \end{center}

Após uma transformação do tempo pelo logaritmo ajustamos um novo
Modelo de Regressão Linear Normal. Novamente, vemos pela análise de
resíduos que o modelo não está bem ajustado e, novamente,
analisando-se o gráfico QQ-plot dos resíduos padronizados pelos
quantis teóricos, tem-se que os pontos não estão bem ajustados sobre
a reta diagonal, apresentando caudas acima da reta. Isso é evidência
de que os erros não apresentam distribuição Normal, e por isso não
seguem a suposição do modelo, de forma que o modelo não está bem
ajustado.

\begin{longtable}[]{@{}lrrrrr@{}}
\toprule
& Df & Sum Sq & Mean Sq & F value & Pr(\textgreater{}F)\tabularnewline
\midrule
\endhead
TI & 1 & 12.0038519 & 12.0038519 & 7.7307831 & 0.0055814\tabularnewline
TII & 1 & 3.8263191 & 3.8263191 & 2.4642459 & 0.1169364\tabularnewline
TIII & 1 & 0.8479272 & 0.8479272 & 0.5460865 & 0.4601804\tabularnewline
TIV & 1 & 0.0819567 & 0.0819567 & 0.0527822 & 0.8183611\tabularnewline
TV & 1 & 8.8407844 & 8.8407844 & 5.6936879 & 0.0173031\tabularnewline
TVI & 1 & 1.6859235 & 1.6859235 & 1.0857772 & 0.2977851\tabularnewline
TVII & 1 & 7.9753725 & 7.9753725 & 5.1363408 & 0.0237485\tabularnewline
TVIII & 1 & 0.2268724 & 0.2268724 & 0.1461115 & 0.7024008\tabularnewline
TIX & 1 & 0.0067131 & 0.0067131 & 0.0043234 & 0.9475944\tabularnewline
Residuals & 669 & 1038.7792302 & 1.5527343 & NA & NA\tabularnewline
\bottomrule
\end{longtable}

\begin{center}\includegraphics{Temas_Constitucionais__2__files/figure-latex/modNormalLog-1} \end{center}

Agora ajustamos um Modelo de Regressão com resposta exponencial (GLM),
no qual observamos, pela análise dos resíduos, que o modelo parece
estar bem ajustado, já que eles se comportam como uma amostra de uma
Distribuição Normal. Observamos que, a um nível de significância
estatística de 5\%, temos que uma ADIN ser de um determinado tema, de
qualquer título que seja, não altera o tempo em que está fora de
pauta, pois nenhum dos p-valores estão abaixo de 0,05. A uma
significância de 10\%, apenas o tema Título I influencia o tempo que a
ADIN permanece fora de pauta, pois seu p-valor é menor do que 0,10.

\begin{longtable}[]{@{}lrrrr@{}}
\toprule
& Estimate & Std. Error & t value &
Pr(\textgreater{}\textbar{}t\textbar{})\tabularnewline
\midrule
\endhead
(Intercept) & 7.6268044 & 0.0925958 & 82.3666403 &
0.0000000\tabularnewline
TI & 0.1974928 & 0.1136747 & 1.7373503 & 0.0827856\tabularnewline
TII & -0.0473660 & 0.0892524 & -0.5306979 & 0.5958043\tabularnewline
TIII & 0.1114243 & 0.0857278 & 1.2997448 & 0.1941361\tabularnewline
TIV & 0.0887070 & 0.0863103 & 1.0277680 & 0.3044304\tabularnewline
TV & 0.0928278 & 0.1662581 & 0.5583354 & 0.5768022\tabularnewline
TVI & -0.0561610 & 0.0974902 & -0.5760683 & 0.5647628\tabularnewline
TVII & 0.1278437 & 0.1549730 & 0.8249418 & 0.4096987\tabularnewline
TVIII & 0.0230261 & 0.1540208 & 0.1494998 & 0.8812043\tabularnewline
TIX & -0.0490168 & 0.2370453 & -0.2067822 & 0.8362429\tabularnewline
\bottomrule
\end{longtable}

\begin{center}\includegraphics{Temas_Constitucionais__2__files/figure-latex/modExp2-1} \end{center}

\begin{verbatim}
## ******************************************************************
##        Summary of the Quantile Residuals
##                            mean   =  0.05667384 
##                        variance   =  0.8752136 
##                coef. of skewness  =  -0.4981755 
##                coef. of kurtosis  =  2.845134 
## Filliben correlation coefficient  =  0.9887146 
## ******************************************************************
\end{verbatim}

Por fim, também ajustamos um Modelo de Regressão com resposta Gama.
Aqui, também observamos, pela análise dos resíduos, que o modelo
parece estar bem ajustado. Assim como para o Modelo Exponencial, a um
nível de significância estatística de 5\%, temos que uma ADIN ser de
um determinado tema, de qualquer título que seja, não altera o tempo
em que está fora de pauta, pois nenhum dos p-valores estão abaixo de
0,05. Ainda, a uma significância de 10\%, apenas o tema Título I
influencia o tempo que a ADIN permanece fora de pauta, pois seu p-valor
é menor do que 0,10.

\begin{longtable}[]{@{}lrrrr@{}}
\toprule
& Estimate & Std. Error & t value &
Pr(\textgreater{}\textbar{}t\textbar{})\tabularnewline
\midrule
\endhead
(Intercept) & 7.6268044 & 0.0862583 & 88.4181852 &
0.0000000\tabularnewline
TI & 0.1974928 & 0.1058945 & 1.8649949 & 0.0626204\tabularnewline
TII & -0.0473660 & 0.0831437 & -0.5696887 & 0.5690804\tabularnewline
TIII & 0.1114243 & 0.0798604 & 1.3952381 & 0.1634077\tabularnewline
TIV & 0.0887070 & 0.0804031 & 1.1032789 & 0.2703033\tabularnewline
TV & 0.0928278 & 0.1548790 & 0.5993568 & 0.5491382\tabularnewline
TVI & -0.0561610 & 0.0908177 & -0.6183925 & 0.5365274\tabularnewline
TVII & 0.1278437 & 0.1443663 & 0.8855509 & 0.3761782\tabularnewline
TVIII & 0.0230261 & 0.1434792 & 0.1604837 & 0.8725486\tabularnewline
TIX & -0.0490168 & 0.2208214 & -0.2219747 & 0.8244014\tabularnewline
(Intercept) & -0.0708972 & 0.0241886 & -2.9310148 &
0.0034940\tabularnewline
\bottomrule
\end{longtable}

\begin{center}\includegraphics{Temas_Constitucionais__2__files/figure-latex/modGam2-1} \end{center}

\begin{verbatim}
## ******************************************************************
##        Summary of the Quantile Residuals
##                            mean   =  0.01260403 
##                        variance   =  1.012228 
##                coef. of skewness  =  -0.4952551 
##                coef. of kurtosis  =  2.840702 
## Filliben correlation coefficient  =  0.9888005 
## ******************************************************************
\end{verbatim}

Concluímos assim que, a uma significância de 5\%, a presença de
nenhum tema influencia o tempo em que uma ADIN permanece fora de pauta.
Já a uma significância de 10\%, apenas a presença do tema Título I
influencia o tempo em que uma ADIN permanece fora de pauta.

\section{Referências}\label{referancias}


\end{document}
